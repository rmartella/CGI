\documentclass[a4paper,11pt]{article}

\usepackage[T1]{fontenc}
\usepackage[utf8]{inputenc}
\usepackage{graphicx}
\usepackage{subfigure}
\usepackage{xcolor}

\renewcommand\familydefault{\sfdefault}
\usepackage{tgheros}
%\usepackage[defaultmono]{droidmono}

\usepackage{amsmath,amssymb,amsthm,textcomp}
\usepackage{enumerate}
\usepackage{multicol}
\usepackage{tikz}

\usepackage{enumitem}
\newlist{legal}{enumerate}{10}
\setlist[legal]{label*=\arabic*.}

\usepackage{geometry}
\geometry{total={210mm,297mm},
left=25mm,right=25mm,%
bindingoffset=0mm, top=20mm,bottom=20mm}


\linespread{1.3}

\newcommand{\linia}{\rule{\linewidth}{0.5pt}}

% custom theorems if needed
\newtheoremstyle{mytheor}
    {1ex}{1ex}{\normalfont}{0pt}{\scshape}{.}{1ex}
    {{\thmname{#1 }}{\thmnumber{#2}}{\thmnote{ (#3)}}}

\theoremstyle{mytheor}
\newtheorem{defi}{Definition}

% my own titles
\makeatletter
\renewcommand{\maketitle}{
\begin{center}
\vspace{2ex}
{\huge \textsc{\@title}}
\vspace{1ex}
\\
\linia\\
\@author \hfill \@date
\vspace{4ex}
\end{center}
}
\makeatother
%%%

% custom footers and headers
\usepackage{fancyhdr}
\pagestyle{fancy}
\lhead{}
\chead{}
\rhead{}
\lfoot{Syllabus-2020-1}
\cfoot{}
\rfoot{Página \thepage}
\renewcommand{\headrulewidth}{0pt}
\renewcommand{\footrulewidth}{0pt}
%

% code listing settings
\usepackage{listings}
\lstset{
    language=Python,
    basicstyle=\ttfamily\small,
    aboveskip={1.0\baselineskip},
    belowskip={1.0\baselineskip},
    columns=fixed,
    extendedchars=true,
    breaklines=true,
    tabsize=4,
    prebreak=\raisebox{0ex}[0ex][0ex]{\ensuremath{\hookleftarrow}},
    frame=lines,
    showtabs=false,
    showspaces=false,
    showstringspaces=false,
    keywordstyle=\color[rgb]{0.627,0.126,0.941},
    commentstyle=\color[rgb]{0.133,0.545,0.133},
    stringstyle=\color[rgb]{01,0,0},
    numbers=left,
    numberstyle=\small,
    stepnumber=1,
    numbersep=10pt,
    captionpos=t,
    escapeinside={\%*}{*)}
}

%%%----------%%%----------%%%----------%%%----------%%%

\begin{document}

\title{CGeIHC \\ Syllabus-2020-1}

\author{M.C. Reynaldo Martell Avila, Grupo \textnumero{2} , Lunes y Miércoles 13:00 - 15:00}

\date{\today}

\maketitle

\section*{Temario}

Introducción a la materia.\textbf{(5 de Agosto)}
\begin{legal}
	\item Introducción a la computación Gráfica. \textbf{(5 y 7 de Agosto)}
	\begin{legal}
		\item Introducción Histórica.
		\begin{legal}
			\item Introducción a la computación gráfica.
			\item Áreas de desarrollo de la Computación Gráfica
			\item Introducción a OpenGL.
		\end{legal}
		\item El Software gráfico \textbf{(7 de Agosto)}
		\begin{legal}
			\item Hardware y software gráfico.
		\end{legal}
	\end{legal}
	\item Pipeline de renderizado. 
	\begin{legal}
		\item Pipeline de OpenGL. \textbf{(12 de Agosto)}
		%\item Presentación de código base. \textbf{(14 de Agosto)}
		\item Transformaciones Geométricas.  \textbf{(14 y 19 de Agosto)}
		\begin{legal}
			\item Coordenadas Homogéneas.
			\item Representación Matricial de transformaciones.
			\item Composición de transformaciones.
		\end{legal}
		\item Espacios coordenados. \textbf{(21 de Agosto)}
		\item Proyecciones. \textbf{(21 y 26 de Agosto)}
		\begin{legal}
			\item Proyección ortogonal.
			\item Proyección en perspectiva.
			\item Cámara sintética.
		\end{legal}
		Presentación de código base \textbf{(26 de Agosto)}
		\item Recorte \textbf{(28 Agosto y 2 de Septiembre)}
		\begin{legal}
			\item Recorte de puntos.
			\item Recorte de Liang Barsky
			\item Recorte de Cohen-Shuterland
		\end{legal}
	\end{legal}
	\item Modelado geométrico y Jerárquico. \textbf{(4 de Septiembre)}
	\\ Presentación de código modelado Geométrico\textbf{(9 Septiembre)}
	\item Dibujo de primitivas en 2D. \textbf{(11 y 18 de Septiembre)}
	\begin{legal}
		\item Algoritmo de Bresenham para líneas.
		\item Algoritmo de Bresenham para circulos.
	\end{legal}
	\textbf{Primer examen parcial. (18 de Septiembre)} \\
	\textbf{Revisión Primer examen parcial. (23 de Septiembre)}
	\item Texturizado \textbf{(25 de Septiembre)}
	\begin{legal}
		\item Filtering.
		\item Wrapping.
		\item Mipmaps.
	\end{legal}
	\item Modelos de color e iluminación. \textbf{(2, 7, 9 y 14 de Octubre)}
	\begin{legal}
		\item Modelos de color.
		\item Iluminación de Gouraud y Phong.
		\item Iluminación tipo de luces.
		\item Iluminación global (Ray tracing).
		\item Presentación de código de Modelos e iluminación.
	\end{legal}
	\item Principios de animación. \textbf{(16, 21, 23, 28 de Octubre)}
	\begin{legal}
		\item Animación por cinemática directa.
		\item Animación por cinemática inversa.
		\item Animación Key Frames.
	\end{legal}
	\item Interfaces de usuarios \textbf{(30 de Octubre, 4, 6, 11 de Noviembre)}
	\begin{legal}
		\item Antecedentes órganos sensoriales.
		\item Realidad Virtual.
		\item Interacción Humano Computadora.
		\item Interfaces de usuarios.
		\item Presentación de código y ejemplos con Kinect.
	\end{legal}
	\textbf{Segundo examen parcial. 13 de Noviembre}
	\\ \textbf{Entrega del proyecto. 27 de Noviembre }
	\\ \textbf{Examen Final. 2 de Diciembre}
	
\end{legal}

\section*{EVALUACIÓN}

\begin{itemize}
	\item Exámenes.	35 \%
	\begin{itemize}
		\item Parcial 1
		\item Parcial 2
	\end{itemize}
	\item Proyecto 35 \%
	\item Laboratorio 20 \%
	\item Tareas e investigaciones 10 \%
\end{itemize}

\section*{EVALUACIÓN}

\begin{itemize}
	\item ANGEL, Edward, Interactive Computer Graphics: A Top-Down Approach
with OpenGL 4, 6ta edition, Portland
Addison-Wesley. 2011.
	\item Alab B. Craig, William R. Sherman, Jeffrey D. Will, Developing Virtual
Reality Applications, Elsevier, 2009
	\item Mario A. Gutiérrez A. Frédéric Vexo, Daniel Thalmann, Stepping into Virtual
Reality, Springer, 2008.
	\item Mark Segal, Kurt Akeley, The OpenGLR Graphics System Version 3.3 (Core
Pro le) The Khronos Group, 2011.
	\item Wilbert O. Galitz, The Essential Guide to User Interface Design, Wiley
Computer Publishing, Second Edition, 2002.
	\item Dave Shreiner, Graham Seliers, John Kessenich, Bill Licea-Kane,
Programming Guide The oficial Guide to Learning
OpenGL Version
4.3, The Khronos Group, Eighth Edition.
	\item David Wolff, OpenGL 4.0 Shading Language Cookbook, Packt publishing,
2011.
\end{itemize}

\end{document}
