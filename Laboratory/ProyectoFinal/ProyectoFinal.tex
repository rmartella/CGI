\documentclass[12pt,letterpaper]{article}
\usepackage{fullpage}
\usepackage[top=2cm, bottom=4.5cm, left=2.5cm, right=2.5cm]{geometry}
\usepackage{amsmath,amsthm,amsfonts,amssymb,amscd}
\usepackage{lastpage}
\usepackage{enumerate}
\usepackage{fancyhdr}
\usepackage{mathrsfs}
\usepackage{xcolor}
\usepackage{graphicx}
\usepackage{listings}
\usepackage{hyperref}

\hypersetup{%
  colorlinks=true,
  linkcolor=blue,
  linkbordercolor={0 0 1}
}
 
\renewcommand\lstlistingname{Algorithm}
\renewcommand\lstlistlistingname{Algorithms}
\def\lstlistingautorefname{Alg.}

\lstdefinestyle{Python}{
    language        = Python,
    frame           = lines, 
    basicstyle      = \footnotesize,
    keywordstyle    = \color{blue},
    stringstyle     = \color{green},
    commentstyle    = \color{red}\ttfamily
}

\setlength{\parindent}{0.0in}
\setlength{\parskip}{0.05in}

% Edit these as appropriate
%\newcommand\course{CSE 3500}
%\newcommand\hwnumber{1}                  % <-- homework number
%\newcommand\NetIDa{netid19823}           % <-- NetID of person #1
%\newcommand\NetIDb{netid12038}           % <-- NetID of person #2 (Comment this line out for problem sets)

\pagestyle{fancyplain}
\headheight 35pt
%\lhead{\NetIDa}
%\lhead{\NetIDa\\\NetIDb}                 % <-- Comment this line out for problem sets (make sure you are person #1)
\chead{\textbf{\Large Proyecto Final\\
Curso: Computaci\'on Gr\'afica e Interacci\'on Humano Computadora, \\ Semestre 2020-1}}
\rhead{\today}
\lfoot{}
\cfoot{}
\rfoot{\small\thepage}
\headsep 5.5em

\begin{document}

\section*{Objetivo:}
\begin{itemize}
\item El alumno presentará en su proyecto final las t\'ecnicas de graficación para la visualizaci\'on de un ambiente tridimensional que consiste en la representaci\'on y animaciones de un parque de diversiones.
\end{itemize}

\textbf{Fecha de Entrega:} 14 de Noviembre del 2019.


\section*{Especificaciones:}

\begin{enumerate}
\item Integre el segmentador de color realizado en el m\'odulo 3 del diplomado con ROS.
\item Modifique la m\'aquina de estados del comportamiento que evade obst\'aculos y se dirige a un destino (objeto de color) del m\'odulo 2 del diplomado utilizando el turtlebot.
\item La localizaci\'on de los objetos destino se realizar\'a por cuadrantes.
\item Utilice ROS para el env\'io de im\'agenes y movimiento del Robot, entre su computadora y el turtlebot.
\item Dado un punto inicial aleatorio el robot deber\'a ser capaz de llegar a dos puntos objetivo (distintos colores) evadiendo obst\'aculos, utilizando m\'aquinas de estados y visi\'on computacional.
\end{enumerate}


\end{document}

