\documentclass[11pt, english]{article}
\usepackage{graphicx}
\usepackage[colorlinks=true, linkcolor=blue]{hyperref}
\usepackage[english]{babel}
\selectlanguage{english}
\usepackage[utf8]{inputenc}
\usepackage[svgnames]{xcolor}
\usepackage{svg}

\usepackage{listings}
\usepackage{afterpage}
\pagestyle{plain}

\definecolor{dkgreen}{rgb}{0,0.6,0}
\definecolor{gray}{rgb}{0.5,0.5,0.5}
\definecolor{mauve}{rgb}{0.58,0,0.82}

\renewcommand{\lstlistingname}{Ejemplo}% Listing -> Ejemplo

\lstdefinestyle{customc}{
  numbers=left,
  belowcaptionskip=1\baselineskip,
  breaklines=true,
  frame=L,
  xleftmargin=\parindent,
  language=C,
  showstringspaces=false,
  basicstyle=\footnotesize\ttfamily,
  keywordstyle=\bfseries\color{green!40!black},
  commentstyle=\itshape\color{purple!40!black},
  identifierstyle=\color{blue},
  stringstyle=\color{orange},
}

%\lstset{language=R,
%    basicstyle=\small\ttfamily,
%   stringstyle=\color{DarkGreen},
%    otherkeywords={0,1,2,3,4,5,6,7,8,9},
%    morekeywords={TRUE,FALSE},
%    deletekeywords={data,frame,length,as,character},
%    keywordstyle=\color{blue},
%    commentstyle=\color{DarkGreen},
%}

\lstset{frame=tb,
language=R,
aboveskip=3mm,
belowskip=3mm,
showstringspaces=false,
columns=flexible,
numbers=none,
keywordstyle=\color{blue},
numberstyle=\tiny\color{gray},
commentstyle=\color{dkgreen},
stringstyle=\color{mauve},
breaklines=true,
breakatwhitespace=true,
tabsize=3
}

\usepackage{here}


\textheight=21cm
\textwidth=17cm
%\topmargin=-1cm
\oddsidemargin=0cm
\parindent=0mm
\pagestyle{plain}

%%%%%%%%%%%%%%%%%%%%%%%%%%
% La siguiente instrucción pone el curso automáticamente%
%%%%%%%%%%%%%%%%%%%%%%%%%%

\usepackage{color}
\usepackage{ragged2e}

\global\let\date\relax
\newcounter{unomenos}
\setcounter{unomenos}{\number\year}
\addtocounter{unomenos}{-1}
\stepcounter{unomenos}
\gdef\@date{ Course \arabic{unomenos}/ 2019}

\begin{document}

\begin{titlepage}

\begin{center}
\vspace*{-1in}
\begin{figure}[htb]
\begin{center}
\centering
\begin{tabular}{@{}cccc@{}}
\includegraphics[width=6cm]{images/EscudoUNAM.png}
\hspace*{1.2in}
\includegraphics[width=6cm]{images/logoIng.png}
\end{tabular}
\end{center}
\end{figure}

FACULTAD DE INGENIERÍA - \@date\\
\vspace*{0.15in}
SECRETARÍA/DIVISIÓN: DIVISIÓN DE INGENIERÍA ELÉCTRICA \\
ÁREA/DEPARTAMENTO: INGENIERÍA EN COMPUTACIÓN \\
\vspace*{0.4in}
\begin{large}
LABORATORIO DE COMPUTACIÓN GRÁFICA E INTERACCIÓN HUMANO COMPUTADORA:\\
\end{large}
\vspace*{0.2in}
\begin{Large}
\textbf{Iluminación y sombreado 1.} \\
\end{Large}
\vspace*{0.3in}
\vspace*{0.3in}
\begin{large}
Reynaldo Martell Avila \\
\end{large}
\vspace*{0.5in}
\vspace*{0.5in}
\begin{large}
\textbf{PRÁCTICA 7} \\
\end{large}
\end{center}
\end{titlepage}

\newcommand{\CC}{C\nolinebreak\hspace{-.05em}\raisebox{.4ex}{\tiny\bf +}\nolinebreak\hspace{-.10em}\raisebox{.4ex}{\tiny\bf +}}
\def\CC{{C\nolinebreak[4]\hspace{-.05em}\raisebox{.4ex}{\tiny\bf ++}}}

\tableofcontents

\newpage
\section{Objetivos de aprendizaje}
\subsection{Objetivos generales:}
El alumno aprenderá a configurar un escenario que hace uso de una fuente de luz
y varios materiales asociados a los objetos.
\subsection{Objetivos específicos:}
\begin{itemize}
\item El alumno comprenderá la importancia de la iluminación en los gráficos por
computadora.
\item El alumno aprenderá a configurar los parámetros para el uso de una fuente de luz,
dependiendo de las características que se desea que tenga.
\item El alumno aprenderá a configurar materiales para modificar la apariencia de los
objetos.
\item El alumno comprenderá la relación entre las características de la fuente de luz y
los materiales ocupados.
\end{itemize}
\section{Recursos a emplear}
\subsection{Software}
Sistema Operativo: Windows 7
Ambiente de Desarrollo: Visual Studio 2017.
\subsection{Equipos}
Los equipos de cómputo con los que cuenta el laboratorio de Computación Gráfica
\subsection{Instrumentos}
\section{Fundamento Teórico}
\begin{itemize}
\item \textbf{Presentación de conceptos.} \\
Las técnicas de Iluminación y sombreado son una forma más realista de hacer los
cálculos de asignación de color en los elementos del escenario.\\
Los cálculos de iluminación son complejos, pero sirven para mejorar el aspecto de
los elementos.\\
Existen diferentes aproximaciones a las características de una fuente luz, entre las
más utilizadas se encuentran las fuentes de luz direccionales, fuentes de luz puntuales y las
fuentes de luz de reflector. Dependiendo de la fuente de luz seleccionada los resultados de
sus rayos de luz afectarán de diferente forma a los objetos en el escenario.\\
La aproximación más utilizada en los gráficos por computadora para el cálculo de
iluminación es el modelo de iluminación de Lambert, el cual divide el comportamiento en
tres componentes: la ambiental, la difusa y la especular.
\item \textbf{Datos necesarios.}
Librería OpenGL 3.3, librería de creación de ventanas, IDE de desarrollo (Visual Studio 2017.
\end{itemize}
\subsection{Desarrollo de actividades}
\begin{enumerate}
\item En el ejemplo de Texturizado agregar en el Shader la condición para descartar
fragmentos transparentes con la imagen que modifico y agrego el canal Alpha.
\item Agregar en el shader texturizado el comportamiento de repetir las texturas.
\item Desarrollar los Shaders para calcular las componentes: ambiental, difusa,
especular.
\item Mostrar la diferencia entre un escenario que utiliza solo colores y el mismo
escenario utilizando fuentes de luz.
\item Modificar los valores de las componentes: ambiental, difusa y especular de la
fuente de luz.
\item Modificar el Shader para agregar materiales a los objetos.
\item Modificar los valores ambientales, difuso y especular de los materiales.

\subsection{Ejercicios}
\section{Observaciones y Conclusiones}
\section{Anexos}
\begin{enumerate}
\item Cuestionario previo.
\begin{enumerate}
\item ¿Cuál es el modelo de iluminación de Phong?.
\item ¿Qué es la iluminación ambiental, difusa y especular?
\item ¿Cómo se calculan la iluminación difusa y especular?
\item ¿Qué es un material en computación gráfica?
\end{enumerate}
\end{enumerate}
\item Actividad de investigación previa.
\begin{enumerate}
\item Agregarle a una imagen con Gimp el canal Alpha a la imagen.
\end{enumerate}
\end{enumerate}

%uoooooooooooooooo tumadreuooooooooooooooooooo UOOOOOOOOOOOOOOOOOOOOOOOOOOOOOOOOOOOOOOOOO
%AL FIN SE TERMINA ESTA PUTA MIERDA!!!!
%USEGREAS OSTOJEOGIRN ojeogiek


\end{document}