\documentclass[11pt, english]{article}
\usepackage{graphicx}
\usepackage[colorlinks=true, linkcolor=blue]{hyperref}
\usepackage[english]{babel}
\selectlanguage{english}
\usepackage[utf8]{inputenc}
\usepackage[svgnames]{xcolor}
\usepackage{svg}

\usepackage{listings}
\usepackage{afterpage}
\pagestyle{plain}

\definecolor{dkgreen}{rgb}{0,0.6,0}
\definecolor{gray}{rgb}{0.5,0.5,0.5}
\definecolor{mauve}{rgb}{0.58,0,0.82}

\renewcommand{\lstlistingname}{Ejemplo}% Listing -> Ejemplo

\lstdefinestyle{customc}{
  numbers=left,
  belowcaptionskip=1\baselineskip,
  breaklines=true,
  frame=L,
  xleftmargin=\parindent,
  language=C,
  showstringspaces=false,
  basicstyle=\footnotesize\ttfamily,
  keywordstyle=\bfseries\color{green!40!black},
  commentstyle=\itshape\color{purple!40!black},
  identifierstyle=\color{blue},
  stringstyle=\color{orange},
}

%\lstset{language=R,
%    basicstyle=\small\ttfamily,
%   stringstyle=\color{DarkGreen},
%    otherkeywords={0,1,2,3,4,5,6,7,8,9},
%    morekeywords={TRUE,FALSE},
%    deletekeywords={data,frame,length,as,character},
%    keywordstyle=\color{blue},
%    commentstyle=\color{DarkGreen},
%}

\lstset{frame=tb,
language=R,
aboveskip=3mm,
belowskip=3mm,
showstringspaces=false,
columns=flexible,
numbers=none,
keywordstyle=\color{blue},
numberstyle=\tiny\color{gray},
commentstyle=\color{dkgreen},
stringstyle=\color{mauve},
breaklines=true,
breakatwhitespace=true,
tabsize=3
}

\usepackage{here}


\textheight=21cm
\textwidth=17cm
%\topmargin=-1cm
\oddsidemargin=0cm
\parindent=0mm
\pagestyle{plain}

%%%%%%%%%%%%%%%%%%%%%%%%%%
% La siguiente instrucción pone el curso automáticamente%
%%%%%%%%%%%%%%%%%%%%%%%%%%

\usepackage{color}
\usepackage{ragged2e}

\global\let\date\relax
\newcounter{unomenos}
\setcounter{unomenos}{\number\year}
\addtocounter{unomenos}{-1}
\stepcounter{unomenos}
\gdef\@date{ Course \arabic{unomenos}/ 2019}

\begin{document}

\begin{titlepage}

\begin{center}
\vspace*{-1in}
\begin{figure}[htb]
\begin{center}
\centering
\begin{tabular}{@{}cccc@{}}
\includegraphics[width=6cm]{images/EscudoUNAM.png}
\hspace*{1.2in}
\includegraphics[width=6cm]{images/logoIng.png}
\end{tabular}
\end{center}
\end{figure}

FACULTAD DE INGENIERÍA - \@date\\
\vspace*{0.15in}
SECRETARÍA/DIVISIÓN: DIVISIÓN DE INGENIERÍA ELÉCTRICA \\
ÁREA/DEPARTAMENTO: INGENIERÍA EN COMPUTACIÓN \\
\vspace*{0.4in}
\begin{large}
LABORATORIO DE COMPUTACIÓN GRÁFICA E INTERACCIÓN HUMANO COMPUTADORA:\\
\end{large}
\vspace*{0.2in}
\begin{Large}
\textbf{Modelado Geómetrico.} \\
\end{Large}
\vspace*{0.3in}
\vspace*{0.3in}
\begin{large}
Reynaldo Martell Avila \\
\end{large}
\vspace*{0.5in}
\vspace*{0.5in}
\begin{large}
\textbf{PRÁCTICA 4} \\
\end{large}
\end{center}
\end{titlepage}

\newcommand{\CC}{C\nolinebreak\hspace{-.05em}\raisebox{.4ex}{\tiny\bf +}\nolinebreak\hspace{-.10em}\raisebox{.4ex}{\tiny\bf +}}
\def\CC{{C\nolinebreak[4]\hspace{-.05em}\raisebox{.4ex}{\tiny\bf ++}}}

\tableofcontents

\newpage
\section{Objetivos de aprendizaje}
\subsection{Objetivos generales:}
El alumno repasará como crear buffers de OpenGL, leer archivos, comprenderá los
diferentes tipos de proyección y las funciones de la librería glm para crear éstas, así como
comprenderá los diferentes sistemas de referencia de OpenGL. Del mismo modo practicará
como colocar en la escena diferentes geometrías.
\subsection{Objetivos específicos:}
El alumno practicará crear geometrías con índices, revisará los sistemas de referencia que
se aplican en OpenGL, comprenderá la utilización de la matriz de modelo, vista, proyección
y la zona de dibujo.
\section{Recursos a emplear}
\subsection{Software}
Sistema Operativo: Windows 7
Ambiente de Desarrollo: Visual Studio 2017.
\subsection{Equipos}
Los equipos de cómputo con los que cuenta el laboratorio de Computación Gráfica
\subsection{Instrumentos}
\section{Fundamento Teórico}
\begin{itemize}
\item \textbf{Presentación de conceptos.} \\
El Modelado Geométrico consiste en la construcción de un modelo a partir de
primitivas, es decir, elementos más sencillo. Un Modelo Geométrico es la representación de
las características geométrica de una entidad concreta o abstracta.
\\
El Modelado Geométrico y el Modelado Jerárquico hace uso de la composición de
operaciones matriciales anidando transformaciones geométricas, por lo cual es importante repasar esto, además de que se explica a grandes rasgos como se forman las primitivas
geométricas.
\item \textbf{Datos necesarios.}
Librería OpenGL 3.3, librería de creación de ventanas, IDE de desarrollo (Visual Studio 2017.
\end{itemize}
\subsection{Desarrollo de actividades}
\begin{enumerate}
\item Ejecutar el código base de la práctica \textbf{04-Modelado Geómetrico}, observar la ejecución.
\item Explicar el código de la Clase \textbf{AbstractModel.h} y sus implementaciones \textbf{Cylinder.cpp}, \textbf{Sphere.cpp} y \textbf{Box.cpp}.
\item  Agregar las cabeceras de los modelos geómetricos Esfera, Caja y Cilindro que se muestran a continuación \ref{list:first}.

\begin{lstlisting}[label={list:first},caption={Inclusión de cabeceras de las formas geómetricas.}, style=customc]
// Model geometric includes
#include "Headers/Sphere.h"
#include "Headers/Cylinder.h"
#include "Headers/Box.h"
\end{lstlisting}

\item Declarar dos objetos de tipo Sphere, Cylinder y Box como se muestra en el ejemplo \ref{list:second}
\begin{lstlisting}[label={list:second},caption={Declarar objetos de tipo esfera, cylindro y caja.}, style=customc]
Sphere sphere1(20, 20);
Cylinder cylinder1(20, 20, 0.5, 0.5);
Box box1;
\end{lstlisting}

\item En el método \textbf{init} agregar las inicializaciones de los buffers de los modelos geómetricos 
\ref{list:third}
\begin{lstlisting}[label={list:third},caption={Inicialización de esfera, cylindro y caja.}, style=customc]
	sphere1.init();
	sphere1.setShader(&shader);
	sphere1.setColor(glm::vec4(0.3, 0.3, 1.0, 1.0));

	cylinder1.init();
	cylinder1.setShader(&shader);
	cylinder1.setColor(glm::vec4(0.3, 0.3, 1.0, 1.0));

	box1.init();
	box1.setShader(&shader);
	box1.setColor(glm::vec4(0.3, 0.3, 1.0, 1.0));
\end{lstlisting}
\item Agregar en el método \textbf{destroy} la eliminación explicita de los objetos, con el fin de liberar espacio de memoria (Buffers y atributos de vertices creados) como se muestra en el ejemplo \ref{list:four}
\begin{lstlisting}[label={list:third},caption={Inicialización de esfera, cylindro y caja.}, style=customc]
	sphere1.destroy();
	cylinder1.destroy();
	box1.destroy();
\end{lstlisting}
\item Agregar en el método \textbf{applicationLoop} despúes de la matriz de modelo el render de una esfera \textbf{sphere1.render(model);} y ejecutar el programa.
\item Agregar enseguida la llamada al método \textbf{sphere1.enableWireMode();}.\\
¿Para que sirve ésta?\\
¿Para que sirve el método \textbf{enableFillMode}?\\
Abrir esta función y reporte que funciones son llamadas internamente. 
\item Colocar otros modelos geómetricos en diferentes posiciones y tamaños.
\item Definir un modelo a construir a partir de primitivas geométricas. El modelo debe
ser en tres dimensiones y consistir de, al menos, seis elementos, de tal forma que
se practique el uso de la composición de operaciones matriciales. Cada alumno
define el modelo a generar.
\end{enumerate}

\subsection{Ejercicios}
\begin{enumerate}
\item Agregue las transformaciones necesarias para que los elementos en pantalla
puedan ser trasladados en el eje X y en el eje Y, al presionar teclas.
\item Agregue las transformaciones necesarias para que los elementos en pantalla
puedan ser manipulados mediante una rotación sobre el eje Y, al presionar alguna
tecla.
\end{enumerate}
\section{Observaciones y Conclusiones}
\section{Anexos}
\begin{enumerate}
\item Cuestionario previo.
\begin{enumerate}
\item ¿Qué es modelado gemétrico?.
\item Investigue el algoritmo para generar una esfera.
\item Investigue el algoritmo para generar un cilindro.
\end{enumerate}
\item Actividad de investigación previa.
\begin{enumerate}
\item Realizar un \textbf{git pull origin master} y un \textbf{git pull myrepo master}, antes de comenzar la práctica.
\end{enumerate}
\end{enumerate}

%uoooooooooooooooo tumadreuooooooooooooooooooo UOOOOOOOOOOOOOOOOOOOOOOOOOOOOOOOOOOOOOOOOO
%AL FIN SE TERMINA ESTA PUTA MIERDA!!!!
%USEGREAS OSTOJEOGIRN ojeogiek


\end{document}