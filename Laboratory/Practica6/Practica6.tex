\documentclass[11pt, english]{article}
\usepackage{graphicx}
\usepackage[colorlinks=true, linkcolor=blue]{hyperref}
\usepackage[english]{babel}
\selectlanguage{english}
\usepackage[utf8]{inputenc}
\usepackage[svgnames]{xcolor}
\usepackage{svg}

\usepackage{listings}
\usepackage{afterpage}
\pagestyle{plain}

\definecolor{dkgreen}{rgb}{0,0.6,0}
\definecolor{gray}{rgb}{0.5,0.5,0.5}
\definecolor{mauve}{rgb}{0.58,0,0.82}

\renewcommand{\lstlistingname}{Ejemplo}% Listing -> Ejemplo

\lstdefinestyle{customc}{
  numbers=left,
  belowcaptionskip=1\baselineskip,
  breaklines=true,
  frame=L,
  xleftmargin=\parindent,
  language=C,
  showstringspaces=false,
  basicstyle=\footnotesize\ttfamily,
  keywordstyle=\bfseries\color{green!40!black},
  commentstyle=\itshape\color{purple!40!black},
  identifierstyle=\color{blue},
  stringstyle=\color{orange},
}

%\lstset{language=R,
%    basicstyle=\small\ttfamily,
%   stringstyle=\color{DarkGreen},
%    otherkeywords={0,1,2,3,4,5,6,7,8,9},
%    morekeywords={TRUE,FALSE},
%    deletekeywords={data,frame,length,as,character},
%    keywordstyle=\color{blue},
%    commentstyle=\color{DarkGreen},
%}

\lstset{frame=tb,
language=R,
aboveskip=3mm,
belowskip=3mm,
showstringspaces=false,
columns=flexible,
numbers=none,
keywordstyle=\color{blue},
numberstyle=\tiny\color{gray},
commentstyle=\color{dkgreen},
stringstyle=\color{mauve},
breaklines=true,
breakatwhitespace=true,
tabsize=3
}

\usepackage{here}


\textheight=21cm
\textwidth=17cm
%\topmargin=-1cm
\oddsidemargin=0cm
\parindent=0mm
\pagestyle{plain}

%%%%%%%%%%%%%%%%%%%%%%%%%%
% La siguiente instrucción pone el curso automáticamente%
%%%%%%%%%%%%%%%%%%%%%%%%%%

\usepackage{color}
\usepackage{ragged2e}

\global\let\date\relax
\newcounter{unomenos}
\setcounter{unomenos}{\number\year}
\addtocounter{unomenos}{-1}
\stepcounter{unomenos}
\gdef\@date{ Course \arabic{unomenos}/ 2019}

\begin{document}

\begin{titlepage}

\begin{center}
\vspace*{-1in}
\begin{figure}[htb]
\begin{center}
\centering
\begin{tabular}{@{}cccc@{}}
\includegraphics[width=6cm]{images/EscudoUNAM.png}
\hspace*{1.2in}
\includegraphics[width=6cm]{images/logoIng.png}
\end{tabular}
\end{center}
\end{figure}

FACULTAD DE INGENIERÍA - \@date\\
\vspace*{0.15in}
SECRETARÍA/DIVISIÓN: DIVISIÓN DE INGENIERÍA ELÉCTRICA \\
ÁREA/DEPARTAMENTO: INGENIERÍA EN COMPUTACIÓN \\
\vspace*{0.4in}
\begin{large}
LABORATORIO DE COMPUTACIÓN GRÁFICA E INTERACCIÓN HUMANO COMPUTADORA:\\
\end{large}
\vspace*{0.2in}
\begin{Large}
\textbf{Texturizado.} \\
\end{Large}
\vspace*{0.3in}
\vspace*{0.3in}
\begin{large}
Reynaldo Martell Avila \\
\end{large}
\vspace*{0.5in}
\vspace*{0.5in}
\begin{large}
\textbf{PRÁCTICA 6} \\
\end{large}
\end{center}
\end{titlepage}

\newcommand{\CC}{C\nolinebreak\hspace{-.05em}\raisebox{.4ex}{\tiny\bf +}\nolinebreak\hspace{-.10em}\raisebox{.4ex}{\tiny\bf +}}
\def\CC{{C\nolinebreak[4]\hspace{-.05em}\raisebox{.4ex}{\tiny\bf ++}}}

\tableofcontents

\newpage
\section{Objetivos de aprendizaje}
\subsection{Objetivos generales:}
El alumno aprenderá el concepto de Texturizado y los pasos necesario para aplicar
esta técnica.
\subsection{Objetivos específicos:}
\begin{itemize}
\item El alumno aprenderá los pasos para realizar el texturizado en superficies planas.
\item El alumno aprenderá a crear texturas y las características que deben tener para su
utilización.
\item El alumno aprenderá los pasos para realizar el texturizado en primitivas
complejas.
\end{itemize}
\section{Recursos a emplear}
\subsection{Software}
Sistema Operativo: Windows 7
Ambiente de Desarrollo: Visual Studio 2017.
\subsection{Equipos}
Los equipos de cómputo con los que cuenta el laboratorio de Computación Gráfica
\subsection{Instrumentos}
\section{Fundamento Teórico}
\begin{itemize}
\item \textbf{Presentación de conceptos.} \\
El texturizado es una técnica para añadir detalle a las geometrías. Sirve para modificar
la apariencia visual de los elementos.
\\
Las texturas deben tener ciertas características para poder ser utilizadas dependiendo
el hardware gráfico utilizado.
\item \textbf{Datos necesarios.}
Librería OpenGL 3.3, librería de creación de ventanas, IDE de desarrollo (Visual Studio 2017.
\end{itemize}
\subsection{Desarrollo de actividades}
\begin{enumerate}
\item Entender la nueva clase para la carga de texturas, así como los pasos necesarios
para poder aplicar dicha textura.
\item Aplicar texturas a un plano.
\item Aplicar textura a un cubo.
\item Aplicar textura a una esfera.
\item Aplicar textura a un cilindro.

\subsection{Ejercicios}
\item Modelar una habitación de la casa de bob sponja con primitivas utilizando texturas.
\section{Observaciones y Conclusiones}
\section{Anexos}
\begin{enumerate}
\item Cuestionario previo.
\begin{itemize}
\item ¿Qué es una Textura en gráficos por computadora?
\item ¿Qué es el mapeo de texturas?
\item ¿Qué es un Texel?
\item ¿Qué es wrapping en gráficos por computadora?
\item ¿Qué es filtering en gráficos por computadora?
\item ¿Qué es un mipmap?
\end{itemize}
\end{enumerate}
\item Actividad de investigación previa.
\begin{enumerate}
\item Instalar el programa gimp 2.10 para Windows.
\end{enumerate}
\end{enumerate}

%uoooooooooooooooo tumadreuooooooooooooooooooo UOOOOOOOOOOOOOOOOOOOOOOOOOOOOOOOOOOOOOOOOO
%AL FIN SE TERMINA ESTA PUTA MIERDA!!!!
%USEGREAS OSTOJEOGIRN ojeogiek


\end{document}